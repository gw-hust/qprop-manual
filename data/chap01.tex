% 第一章
\section{编译 $Qprop$}

$Qprop$ 官网 \url{http://qprop.de},提供的源代码只能每次计算是重新编译二进制软件,不能复用。\url{https://github.com/jam31118/rigged-qprop}提供的版本改善了软件的复用性,只要不涉及到修改吸收势函数、修改原子,一次编译得到的二进制文件就可以多次使用。

本文以 github 上的$Qprop$版本蓝本(v2),在已有配置中新增两个参数 \textit{my-delay}、\textit{dual-m}分别控制圆偏光中激光束的时间延迟与初始 $p$ 态是否为 $m= \pm 1$的混合情况。代码地址 \url{https://github.com/rachpt/rigged-qprop}。

英文使用手册 \url{https://arxiv.org/pdf/1603.05529.pdf}.

\subsection{环境搭建}
GNU/Linux 操作系统,如果是还需要作为主力系统使用,推荐使用 \href{https://manjaro.org/download/}{Manjaro Linux}(KDE、XFCE版本)、\href{https://linuxmint.com/download.php}{linuxmint}(Cinnamon版本);不推荐使用 Ubuntu,因为在长时间高强度的计算负载下, Ubuntu 系统的桌面环境极易出现问题;也不推荐使用 Windows 10 的 Linux 子系统(WSL),高强度的 CPU 占用可能会让 Windows 直接挂掉。


下面提供 Manjaro 与 Linuxmint 系统下编译 $Qprop$ 软件的详细步骤与注意事项。

\begin{enumerate}
    \item 安装操作系统。准备至少 500 GB 硬盘,容量不小于4 GB U盘,到Linux 发行版官网或者 \href{https://mirrors.tuna.tsinghua.edu.cn/}{清华大学镜像}、\href{https://mirrors.ustc.edu.cn/}{中科大镜像}站点下载系统 iso 镜像,使用 \url{https://rufus.ie/} 制作 USB 启动盘。
    \item 完成一些基础设置,比如修改系统更新源为国内镜像站点,安装中文输入法,更新软件仓库,安装 gcc、make、git axel 等软件。
    \item 使用 git clone 代码仓库,或者下载源代码压缩包,解压到一个特定的工作目录(以后不能修改该目录名)。
    \item 安装 $Qprop$ 所需要的依赖库,gsl(GNU Scientific Library),openmpi(并行计算),boost。
    \item 在 $Qprop$ src 目录下执行 make 命令,编译源代码,二进制文件在 bin 目录下。
    \item 复制并进入计算配置文件夹,打开终端,按顺序依次执行二进制文件,进行计算,测试软件是否正常。
\end{enumerate}

\subsection{依赖安装}
$Qprop$ 的依赖库是独立于系统依赖库单独存在的,需要自己编译特定版本的依赖库。

特别注意与说明:

2. gcc 版本可以使用 5.x (manjaro) 与 7.x (linuxmint),根据我的测试较新的 8.x 乃至 9.x 虽然能够正常编译 $Qprop$,但是在运行 real-prop 过程时会立即报错: \ldots 核心已存储 \ldots \\
Linuxmint 安装依赖命令:
\begin{lstlisting}[language=bash]
sudo apt update
sudo apt install gcc-5 g++-5 autoconf make git wget axel htop
# 查看版本信息
gcc   --version
gcc-5 --version
\end{lstlisting}
Manjaro 安装依赖命令:
\begin{lstlisting}[language=bash]
sudo pacman -Syu
sudo pacman -S yay make git wget axel htop
sudo yay -S gcc5
# 查看版本信息
gcc   --version
gcc-5 --version
\end{lstlisting}

安装 gsl、,openmpi、boost:
\begin{lstlisting}[language=bash]
export QPROP_HOME=/path/to/root/directory/of/qprop
export QPROP_DEP_DIR=/path/to/root/directory/of/qprop/dep
# 比如我的
export QPROP_HOME=/home/rachpt/rigged-qprop
export QPROP_DEP_DIR=/home/rachpt/rigged-qprop/dep
# 安装依赖库
bash $QPROP_HOME/prereq/install.sh
\end{lstlisting}

如果因为校园网连接德国的网站慢,可能会因为超时而安装失败,这是需要修改一下下面几个文件。\\
\filename{src/prereq/script/install-gsl.sh}
\begin{lstlisting}
# 第 46 行 wget 改成 axel -n 10 -4 
cd $SRC_DOWN_DIR
wget $SRC_URL
if [ "$?" -ne "0" ]
then
# --------改成-----------
cd $SRC_DOWN_DIR
axel -n 10 -4 $SRC_URL
if [ "$?" -ne "0" ]
then
# 说明: -n 10 表示 10 线程下载,-4 表示使用 ipv4
\end{lstlisting}
\filename{src/prereq/script/install-openmpi.sh}、\filename{src/prereq/script/install-boost.sh}
\begin{lstlisting}
cd $SRC_DOWN_DIR

curl -LO "$SRC_URL"

if [ "$?" -ne "0" ]
then
# -------改成----------
cd $SRC_DOWN_DIR

axel -n 10 -4 $SRC_URL

if [ "$?" -ne "0" ]
then
# 说明:curl 改成 axel,并指定使用 ipv4 地址
\end{lstlisting}


\subsection{编译}
切换到 \filename{src/main/} 目录,打开终端(bash),并输入下列命令,确保环境变量 $QPROP\_HOME$、$QPROP\_DEP\_DIR$ 被正确设置:
\begin{lstlisting}[language=bash]
export QPROP_HOME=/path/to/root/directory/of/qprop
export QPROP_DEP_DIR=/path/to/root/directory/of/qprop/dep
\end{lstlisting}

补充 bash 编程基础:等号两边不能有空格,路径别用包含特殊字符(比如空格)的路径。

\begin{lstlisting}[language=bash]
cd /path/to/qprop/src/main

# 使用 8  线程编译
make -j 8 CXX=gcc-5

\end{lstlisting}

如有需要,可以修改 \filename{/path/to/qprop/src/main/makefile} 头部两行,其中 $QPROP\_BIN$ 表示编译二进制文件安装路径,在每次修改势函数后可以修改该值为新的文件夹。


如果一切正常,没有报错,会在 \filename{/path/to/qprop/bin}(bin 与 $QPROP\_BIN$ 一致) 得到编译好的二进制文件:eval-tsurff、eval-tsurff-mpi、imag-prop、real-prop、ppp、ppp-mpi。其中有 mpi 后缀的表示是并行运算程序。主要使用 imag-prop、real-prop、ppp-mpi、eval-tsurff-mpi 这四个二进制文件程序。


\iffalse
bb \verb|\usepackage{amsmath}| aa

\begin{lstlisting}[language=Python]
import numpy as np
    
def incmatrix(genl1,genl2):
    m = len(genl1)
    n = len(genl2)
    M = None #to become the incidence matrix
    VT = np.zeros((n*m,1), int)  #dummy variable
    # 注释 
    #compute the bitwise xor matrix
    M1 = bitxormatrix(genl1)
    M2 = np.triu(bitxormatrix(genl2),1) 

    for i in range(m-1):
        for j in range(i+1, m):
            [r,c] = np.where(M2 == M1[i,j])
            for k in range(len(r)):
                VT[(i)*n + r[k]] = 1;
                VT[(i)*n + c[k]] = 1;
                VT[(j)*n + r[k]] = 1;
                VT[(j)*n + c[k]] = 1;
                
                if M is None:
                    M = np.copy(VT)
                else:
                    M = np.concatenate((M, VT), 1)
                
                VT = np.zeros((n*m,1), int)
    
    return M
\end{lstlisting}

\begin{lstlisting}[language=bash]
#!/usr/bin/env bash

echo 'ok'
\end{lstlisting}


\begin{verbatim}
    Text enclosed inside \texttt{verbatim} environment 
    is printed directly 
    and all \LaTeX{} commands are ignored.
\end{verbatim}

\begin{verbatim*}
    Text enclosed inside \texttt{verbatim} environment 
    is printed directly 
    and all \LaTeX{} commands are ignored.
\end{verbatim*}


In the directory \verb|C:\Windows\system32| you can find a lot of Windows 
system applications. 
 
The \verb+\ldots+ command produces \ldots
\fi
