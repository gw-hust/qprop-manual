% 第二章
\section{开始使用}

这里以演示 \filename{/path/to/qprop/src/example/attoclock/} 示例的运行。

\begin{lstlisting}[language=bash]
# 新建工作文件夹
mkdir -p /path/to/qprop/jobs/test-jobs

# 复制示例参数文件
cp -r /path/to/qprop/src/example/attoclock /path/to/qprop/jobs/test-jobs

# 进入参数所在文件夹
cd /path/to/qprop/jobs/test-jobs/attoclock

# 设置 QPROP 环境变量,注意使用英文双引号
# set qprop home
export QPROP_HOME="/path/to/qprop"
export QPROP_DEP_DIR="$QPROP_HOME/dep"
# other settings
export LD_LIBRARY_PATH=$LD_LIBRARY_PATH:$QPROP_DEP_DIR/openmpi/lib
export LD_LIBRARY_PATH=$LD_LIBRARY_PATH:$QPROP_DEP_DIR/gsl/lib

## 运行
# 第一步
/path/to/qprop/bin/imag-prop

# 待第一步结束后,运行第二步
/path/to/qprop/bin/real-prop

# 待第二步结束后,运行第三步
"$QPROP_DEP_DIR/openmpi/bin/mpiexec" -n 8 "$QPROP_HOME/bin/eval-tsurff-mpi"
\end{lstlisting}

说明:第三步 -n 8 表示使用 8 线程并行运算。如果 \ref{use_ppp} 处设置了使用 use-ppp,那么在第二第三不之间还有一个 ppp 的并行运算。

\begin{lstlisting}[language=bash]
"$QPROP_DEP_DIR/openmpi/bin/mpiexec" -n 8 "$QPROP_HOME/bin/ppp-mpi"
\end{lstlisting}

\newpage
% \vspace{2em}
一个经过优化后的 命令导入文件\filename{include.sh}:

\lstinputlisting[language=bash]{code/include.sh}