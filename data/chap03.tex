% 第三章
\section{$Qprop$ 计算参数}

参数文件有一下三个:\filename{init.param},\filename{propagate.param},\filename{tsurff.param}。参数文件为纯文本文件,utf-8 编码,Unix 格式换行风格。
其中使用 英文井号(\#)作为注释的开始,单行注释。

\subsection{init.param}
典型的 init 参数配置如下:
\begin{lstlisting}[language=bash]
nuclear-charge double 1.0
pot-cutoff double 25.0
delta-r double 0.2
radial-grid-size double 100.0
ell-grid-size long 2
qprop-dim long 44
initial-m long -1
initial-ell long 1
# control m = \pm 1
dual-m bool 0
# \alpha --> 势能
effpot-alpha double 2.40694033726
\end{lstlisting}

参数意义:
\begin{table}[htbp]
    \begin{tabular}{c|c|l}
    \hline
    参数 & 一般值 & 意义 \\
    \hline
    nuclear-charge & 1.0 & 原子序号,用于指定计算原子 \\
    pot-cutoff & 25.0 & 截断能大小(原子单位),对应 $R_{co}$ \\
    delta-r & 0.2 & 最小步长(原子单位) \\
    radial-grid-size & 100.0 & 网格大小 \\
    qprop-dim & 34/44 & 计算模式(34 线偏、44 圆偏) \\
    initial-m & 0、$\pm$ 1 & 原子磁量子数 m \\
    initial-ell & 0,1、\dots & 轨道量子数 \\
    dual-m & 0/1 & 控制是否为混态(m = $\pm$ 1) \\
    effpot-alpha & -- & 非氢原子 吸收势修正量,使用配套 python 脚本计算 \\
    \hline
    \end{tabular}
\end{table}


\subsection{propagate.param}
圆偏光情况(qprop-dim = 44):

\begin{lstlisting}[language=bash]
imag-width double 150.0
ell-grid-size long 25
delta-t double 0.05

# laser 1  100nm  5x10^14  8oc
max-electric-field-x-1 double 0.02669007
omega-x-1 double 0.11390838
num-cycles-x-1 double 8.0
phase-pi-x-1 double 0.0
my-delay-x-1 double 0.0

max-electric-field-y-1 double 0.02669007
omega-y-1 double 0.11390838
num-cycles-y-1 double 8.0
phase-pi-y-1 double 0.5*M_PI
my-delay-y-1 double 0.0
\end{lstlisting}

-1 表示第一束激光,类似的 -2 表示第二束激光。如有需要使用线偏与圆偏混合模式,修改其中一束光的 $x$ 或者 $y$ 方向的电场强度为 0 即可。


参数意义:
\begin{table}[htbp]
    \begin{tabular}{c|c|l}
    \hline
    参数 & 一般值 & 意义 \\
    \hline
    imag-width & 150.0 & 虚势不等于零的区域的宽度 \\
    ell-grid-size & $\sqrt{2 \times 10 U_p}$ & 球谐波展开角动量最大量子数 \\
    delta-t & delta-r/4 = 0.05 & 演化时间步长(原子单位) \\
    max-electric-field-x-1 & 0.02669(5e14) & 第一束激光$x$方向电场最大值 \\
    omega-y-1 & 0.1139 & 第一束激光$y$方向角频率 \\
    num-cycles-x-1 & 8 & 第一束激光$x$方向持续光周期数量 \\
    phase-pi-y-1 & 0.5*M\_PI & 第一束激光$y$方向 cep 值 \\
    my-delay-y-1 & 0 & 可选参数,激光延时 \\
    \hline
    \end{tabular}
\end{table}

线偏光情况(qprop-dim = 34):

\begin{lstlisting}[language=bash]
# width of the region where imaginary potential is different from zero
imag-width double 150.0
# maximum number of ells in the expansion of the wave function used during time propagation
ell-grid-size long 600
# time step for propagation; delta-r/4 is a sensible choice
delta-t double 0.0375

# maximum value of the electric field of the laser pulse
max-electric-field double 5.3379853388e-02
# frequency of the laser pulse
omega double 2.2800000000e-02
num-cycles double 6.0
\end{lstlisting}

如果需要使用多束线偏振激光,类似与圆偏光情况,使用 $omega\mbox{-}z\mbox{-}1$、$omega\mbox{-}z\mbox{-}2$ 的形式指定各光场形式。


参数意义:
\begin{table}[htbp]
    \begin{tabular}{c|c|l}
    \hline
    参数 & 一般值 & 意义 \\
    \hline
    imag-width & 150.0 & 虚势不等于零的区域的宽度 \\
    ell-grid-size & $\sqrt{2 \times 10 U_p}$ & 球谐波展开角动量最大量子数 \\
    delta-t & delta-r/4 = 0.05 & 演化时间步长(原子单位) \\
    max-electric-field & 0.02669(5e14) & 激光电场最大值 \\
    omega & 0.1139(400 nm) & 激光角频率 \\
    num-cycles & 8 & 激光持续光周期数量 \\
    \hline
    \end{tabular}
\end{table}


\subsection{tsurff.param} \label{use_ppp}


\begin{lstlisting}[language=bash]
# distance to the t-SURFF boundary
# 一般取 4 * pot-cutoff,4×R_{co}
R-tsurff double 100.0

# This determines the duration of the simulation (slowest electron to reach the t-SURFF boundary).
p-min-tsurff double 0.2

# largest k value in the calculated spectrum
k-max-surff double 0.9

# number of k values for the spectrum
num-k-surff long 500

# number of angles theta (\theta \in [0:\pi])
# N_{\theta} < 3 时,N_{\theta} = 3
num-theta-surff long 3

# number of angles phi (\phi \in [0:2\pi))
num-phi-surff long 60

# delta-k-scheme=1: equidistant k grid discretization w.r.t momentum; delta-k-scheme=2: equidistant k grid discretization w.r.t energy
delta-k-scheme long 2

# expansion-scheme=2: true expansion of the spectrum in spherical harmonics
# expansion-scheme=1: use for small number of angles and large number of ells (no partial spectra are produced)
expansion-scheme long 1

# how many time steps are processed at a time during evaluation of the spectrum
cache-size-t long 512

# PPP configuration
use-ppp bool 0
\end{lstlisting}


一般需要修改 $p\mbox{-}min\mbox{-}tsurff$、$k\mbox{-}max\mbox{-}surff$ 至合适的范围,特别是 $p\mbox{-}min\mbox{-}tsurff$,该值越大,计算量越小,$p\mbox{-}min\mbox{-}tsurff$ 贡献在计算量的分母上。

$use\mbox{-}ppp$ 用于控制第二步 real-prop 的后部分外场作用结束后是否使用并行加速运算。使用 ppp 需要时注意,此时计算过程文件中得到的电离率小于实际电离率。并且在传统的三步计算过程中的第二三步之间多一个 ppp 的运行步骤。
\\

\textbf{注意:}
\textcolor{red}{每次计算中不能修改三个参数文件,修改后需要重新重头计算。}
